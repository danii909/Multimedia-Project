% =================================================
% CAPITOLO 1 – Introduzione
% =================================================

\chapter{Introduzione}

\section{Contesto e Motivazioni}

La crescente diffusione di dispositivi mobili dotati di fotocamere ad alta risoluzione ha reso la produzione di contenuti video estremamente accessibile. Tuttavia, la registrazione manuale (hand-held) introduce inevitabilmente instabilità dovute a micro-movimenti involontari dell'operatore, vibrazioni ambientali o oscillazioni meccaniche del supporto.

Tali instabilità si manifestano sotto forma di jitter, oscillazioni improvvise, drift progressivo o rotazioni indesiderate, compromettendo la qualità percettiva del video e rendendo difficoltosa la fruizione del contenuto.

La stabilizzazione video digitale si propone di correggere tali effetti attraverso tecniche di stima del moto e compensazione geometrica, intervenendo direttamente sulle trasformazioni tra frame consecutivi. A differenza dei sistemi hardware (gimbal o stabilizzatori ottici), le tecniche software operano a posteriori sul segnale acquisito e richiedono algoritmi robusti di analisi e filtraggio del moto.

Nel contesto dei sistemi multimediali, la stabilizzazione rappresenta un problema multidisciplinare che coinvolge:
\begin{itemize}
    \item visione artificiale,
    \item elaborazione di segnali discreti,
    \item modellazione geometrica delle trasformazioni,
    \item ottimizzazione numerica.
\end{itemize}

Il presente progetto si inserisce in tale ambito proponendo un sistema modulare di stabilizzazione video implementato in Python, con interfaccia grafica interattiva e possibilità di confronto tra differenti metodologie di stima del moto.

\section{Obiettivi del Progetto}

L'obiettivo principale del progetto è la realizzazione di un sistema completo di stabilizzazione video che consenta:

\begin{itemize}
    \item l'implementazione di più algoritmi di stima del moto globale,
    \item la comparazione quantitativa tra differenti approcci,
    \item la configurabilità dinamica dei parametri algoritmici,
    \item la visualizzazione grafica delle traiettorie e delle metriche di stabilità,
    \item l'analisi comparativa dei risultati ottenuti.
\end{itemize}

In particolare, il sistema è stato progettato con una forte separazione tra:
\begin{enumerate}
    \item logica di elaborazione,
    \item sistema di configurazione,
    \item interfaccia utente,
    \item modulo di analisi e visualizzazione.
\end{enumerate}

Un ulteriore obiettivo riguarda la valutazione oggettiva delle prestazioni tramite metriche quantitative, quali RMS dello spostamento e percentuale di riduzione del jitter.

\section{Organizzazione della Relazione}

La relazione è strutturata come segue:

\begin{itemize}
    \item Il Capitolo 2 introduce i fondamenti teorici della stabilizzazione video: modelli di trasformazione, costruzione della traiettoria e filtraggio.
    \item Il Capitolo 3 descrive nel dettaglio i tre metodi implementati per la stima del moto globale.
    \item Il Capitolo 4 illustra l'architettura software modulare a cinque livelli del sistema.
    \item Il Capitolo 5 approfondisce l'implementazione tecnica, con listati di codice e descrizione delle metriche calcolate.
    \item Il Capitolo 6 descrive l'interfaccia utente Streamlit e le funzionalità offerte.
    \item Il Capitolo 7 riporta i test sperimentali e l'analisi quantitativa dei risultati.
    \item Il Capitolo 8 conclude il lavoro evidenziando i risultati principali, le limitazioni e i possibili sviluppi futuri.
\end{itemize}
