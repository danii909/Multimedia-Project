\chapter*{Abstract}
\addcontentsline{toc}{chapter}{Abstract}

Il presente lavoro descrive la progettazione e l'implementazione di un sistema modulare di stabilizzazione video digitale sviluppato in Python. Il sistema integra tre metodologie di stima del moto globale: Block-Based Motion Estimation tramite Block Matching con metrica SAD/MAD, Optical Flow basato su corner detection Shi-Tomasi e algoritmo di Lucas-Kanade, e Feature Matching tramite descrittori ORB con Brute-Force Matcher e Lowe's ratio test. Per ciascun metodo feature-based sono supportati tre modelli di trasformazione: similitudine rigida a 4 parametri (\emph{Partial}), trasformazione affine completa a 6 parametri e omografia proiettiva a 8 parametri, tutti stimati tramite RANSAC. Il filtraggio della traiettoria cumulativa è implementato con tre strategie: media mobile centrata, filtro gaussiano pesato e filtro esponenziale IIR. I test sperimentali condotti su una sequenza di 573 frame a 1080p mostrano che la variante ORB con modello Partial raggiunge la riduzione di jitter più elevata (87.7\% sull'asse X, 92.8\% sull'asse Y), mentre Optical Flow Partial offre il miglior compromesso tra qualità e tempo di elaborazione (18.1\,s totali). Il sistema è corredato da un'interfaccia web interattiva sviluppata con Streamlit che consente stabilizzazione singola, confronto multi-metodo sincronizzato e analisi quantitativa delle metriche di qualità.
