% =================================================
% CAPITOLO 6 – Interfaccia Utente
% =================================================

\chapter{Interfaccia Utente}

\section{Struttura Generale dell'Applicazione}

L'applicazione è stata sviluppata utilizzando il framework Streamlit, che consente la creazione di interfacce web interattive direttamente in Python.

La struttura dell'interfaccia è organizzata in:

\begin{itemize}
    \item una \textbf{sidebar} dedicata al caricamento del video,
    \item tre \textbf{tab principali} per stabilizzazione, confronto e analisi,
    \item componenti dinamici per la visualizzazione di risultati e metriche.
\end{itemize}

L'approccio adottato separa chiaramente la raccolta dei parametri dalla logica di esecuzione, mantenendo il codice dell'interfaccia indipendente dagli algoritmi di stabilizzazione.

\section{Caricamento del Video}

La sidebar consente all'utente di caricare un file video nei formati più comuni (MP4, AVI, MOV, MKV).

Una volta caricato il file:

\begin{enumerate}
    \item il video viene salvato temporaneamente,
    \item vengono estratte informazioni quali:
    \begin{itemize}
        \item risoluzione,
        \item numero di frame,
        \item durata,
        \item frame rate,
    \end{itemize}
    \item i dati vengono memorizzati nello stato della sessione.
\end{enumerate}

La gestione dello stato evita la ricostruzione del file temporaneo a ogni aggiornamento dell'interfaccia, migliorando l'efficienza.

\section{Stabilizzazione Singola (Confronto 1v1)}

Il primo tab consente di selezionare un singolo algoritmo e confrontarlo direttamente con il video originale.

Il flusso operativo è il seguente:

\begin{enumerate}
    \item selezione della famiglia di algoritmo,
    \item scelta del modello di trasformazione,
    \item configurazione dei parametri,
    \item avvio dell'elaborazione.
\end{enumerate}

Al termine della stabilizzazione, l'interfaccia mostra:

\begin{itemize}
    \item video comparativo sincronizzato,
    \item metriche quantitative,
    \item grafici della traiettoria,
    \item possibilità di download dei risultati.
\end{itemize}

Questo approccio consente un'analisi qualitativa immediata dell'effetto della stabilizzazione.

\section{Confronto Multi-Metodo}

Il secondo tab permette l'esecuzione simultanea di più algoritmi sulla stessa sequenza video.

L'utente può selezionare più metodi tramite checkbox e configurare parametri comuni e specifici.

Durante l'elaborazione vengono mostrate:

\begin{itemize}
    \item barra di progresso globale,
    \item barra di progresso per ciascun algoritmo,
    \item stato di completamento.
\end{itemize}

Al termine dell'elaborazione è possibile:

\begin{itemize}
    \item visualizzare i video stabilizzati affiancati,
    \item generare una griglia comparativa sincronizzata,
    \item analizzare metriche aggregate.
\end{itemize}

Questo modulo rappresenta il cuore comparativo del sistema.

\section{Analisi delle Metriche}

Il terzo tab è dedicato alla visualizzazione quantitativa dei risultati.

Le funzionalità principali includono:

\begin{itemize}
    \item tabella riassuntiva con RMS e tempo di elaborazione,
    \item grafici comparativi tra metodi,
    \item visualizzazione delle traiettorie raw e smoothed,
    \item esportazione delle metriche in formato JSON.
\end{itemize}

La rappresentazione grafica facilita l'interpretazione dei risultati e consente di confrontare in modo oggettivo le performance dei diversi approcci.

\section{Visualizzazione Grafica}

Il sistema integra grafici dinamici generati tramite librerie di plotting.

Le principali visualizzazioni includono:

\begin{itemize}
    \item traiettoria cumulativa originale e filtrata,
    \item confronto RMS tra metodi,
    \item confronto percentuale di riduzione del jitter.
\end{itemize}

Le visualizzazioni sono generate dinamicamente a partire dalle metriche calcolate durante l'elaborazione.

\section{Esperienza Utente}

Particolare attenzione è stata dedicata all'usabilità:

\begin{itemize}
    \item layout a colonne per confronto visivo immediato,
    \item barre di progresso informative,
    \item feedback visivo al completamento,
    \item possibilità di download diretto dei risultati.
\end{itemize}

È stato inoltre applicato uno stile grafico personalizzato per migliorare leggibilità e coerenza visiva dell'applicazione.
