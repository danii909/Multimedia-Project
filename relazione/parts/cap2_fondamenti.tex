% =================================================
% CAPITOLO 2 – Fondamenti Teorici
% =================================================

\chapter{Fondamenti Teorici}

\section{Il Problema della Stabilizzazione Video}

Un video può essere modellato come una sequenza discreta di frame
\[
\{I_t\}_{t=1}^{N}
\]
dove $I_t$ rappresenta l'immagine al tempo discreto $t$.

In condizioni ideali, la traiettoria della telecamera dovrebbe essere fluida e coerente con il movimento intenzionale dell'operatore. Tuttavia, durante la registrazione manuale, si introducono variazioni indesiderate che si manifestano come:

\begin{itemize}
    \item \textbf{Jitter}: oscillazioni rapide e ad alta frequenza,
    \item \textbf{Drift}: spostamento progressivo non intenzionale,
    \item \textbf{Micro-rotazioni}: variazioni angolari tra frame consecutivi,
    \item \textbf{Vibrazioni meccaniche}: tipiche di dispositivi montati su veicoli.
\end{itemize}

La stabilizzazione digitale mira a stimare il moto globale tra frame consecutivi e a rimuovere la componente ad alta frequenza responsabile dell'instabilità, preservando al contempo il movimento intenzionale a bassa frequenza.

\subsection{Tipologie di Instabilità}

Le principali categorie di instabilità che affliggono le riprese manuali sono:
\begin{itemize}
    \item \textbf{Jitter ad alta frequenza}: oscillazioni rapide con ampiezza ridotta (tipicamente 2–15\,px) causate dal tremore della mano. Si manifestano come flickering visivo a frequenze superiori a circa 5\,Hz.
    \item \textbf{Drift a bassa frequenza}: deriva progressiva della traiettoria causata da movimenti posturali dell'operatore. Può accumularsi su scale temporali di diversi secondi.
    \item \textbf{Oscillazioni periodiche}: pattern ritmici legati a movimenti ciclici (es.\ passi durante la camminata). Producono uno spostamento oscillante prevalentemente verticale con periodo proporzionale alla frequenza del passo.
    \item \textbf{Micro-rotazioni} (roll): variazioni angolari involontarie attorno all'asse ottico, particolarmente evidenti nei dispositivi mobili tenuti con una mano.
    \item \textbf{Vibrazioni meccaniche}: instabilità ad alta frequenza tipiche di riprese da veicoli o droni, spesso accompagnate da componenti di shake multi-assiale.
\end{itemize}

\section{Modelli di Movimento Globale}

La stabilizzazione si basa sull'assunzione che il moto dominante tra due frame consecutivi possa essere approssimato tramite una trasformazione geometrica globale.

\subsection{Modello di Traslazione}

Il modello più semplice considera uno spostamento bidimensionale:

\[
\begin{bmatrix}
x' \\
y'
\end{bmatrix}
=
\begin{bmatrix}
x \\
y
\end{bmatrix}
+
\begin{bmatrix}
\Delta x \\
\Delta y
\end{bmatrix}
\]

Questo modello è computazionalmente efficiente ma non tiene conto di rotazioni o deformazioni prospettiche.

\subsection{Trasformazione di Similarità (Affine Parziale)}

La trasformazione di similarità, nota anche come \emph{affine parziale}, estende il modello traslazionale includendo rotazione e scala uniforme:

\[
\begin{bmatrix}
x' \\
y' \\
1
\end{bmatrix}
=
\begin{bmatrix}
s\cos\theta & -s\sin\theta & t_x \\
s\sin\theta &  s\cos\theta & t_y \\
0 & 0 & 1
\end{bmatrix}
\begin{bmatrix}
x \\
y \\
1
\end{bmatrix}
\]

Questo modello a 4 parametri $(t_x, t_y, \theta, s)$ è la scelta di default nel sistema (\texttt{transform\_type: partial}, stimato tramite \texttt{estimateAffinePartial2D}). Rispetto alla traslazione pura cattura anche le micro-rotazioni dell'asse ottico, mantenendo però maggiore stabilità numerica rispetto al modello affine completo in quanto impone la conservazione degli angoli (no shear) e una scala isotropa.

\subsection{Trasformazione Affine}

La trasformazione affine permette di modellare traslazione, rotazione, scala e shear:

\[
\begin{bmatrix}
x' \\
y' \\
1
\end{bmatrix}
=
\begin{bmatrix}
a_{11} & a_{12} & t_x \\
a_{21} & a_{22} & t_y \\
0 & 0 & 1
\end{bmatrix}
\begin{bmatrix}
x \\
y \\
1
\end{bmatrix}
\]

Essa rappresenta un compromesso tra flessibilità e stabilità numerica.

\subsection{Omografia}

Nel caso più generale si utilizza una trasformazione proiettiva~\cite{hartley2003multiple}:

\[
\mathbf{x}' = H \mathbf{x}
\]

dove $H$ è una matrice $3 \times 3$ definita a scala, in grado di modellare variazioni prospettiche.

Questo modello è più espressivo ma più sensibile al rumore e agli errori di stima.

\section{Stima del Moto Globale}

La stima del moto globale tra frame consecutivi può avvenire secondo due principali paradigmi:

\begin{itemize}
    \item \textbf{Approccio block-based}, basato su matching tra blocchi di pixel;
    \item \textbf{Approccio feature-based}, basato su rilevamento e tracciamento di punti di interesse.
\end{itemize}

Nel primo caso, l'immagine viene suddivisa in blocchi regolari e si ricerca lo spostamento ottimale tramite metriche di similarità.

Nel secondo caso, si individuano feature robuste e si stimano i parametri della trasformazione tramite metodi robusti (es. RANSAC).

\section{Costruzione della Traiettoria}

Una volta stimati i parametri di trasformazione tra frame consecutivi, si ottiene una sequenza di trasformazioni incrementali:

\[
T_1, T_2, \dots, T_N
\]

La traiettoria cumulativa della camera viene costruita tramite composizione:

\[
C_t = T_1 \circ T_2 \circ \dots \circ T_t
\]

Nel caso del modello traslazionale, la traiettoria si riduce alla somma cumulativa degli spostamenti:

\[
X_t = \sum_{i=1}^{t} \Delta x_i
\quad
Y_t = \sum_{i=1}^{t} \Delta y_i
\]

La traiettoria risultante contiene sia il movimento desiderato sia la componente ad alta frequenza responsabile dell'instabilità.

\section{Filtraggio della Traiettoria}

Per separare il movimento intenzionale dal jitter, si applica un filtro passa-basso alla traiettoria.

Sia $C_t$ la traiettoria cumulativa e $\tilde{C}_t$ la versione filtrata:

\[
\tilde{C}_t = \mathcal{F}(C_t)
\]

dove $\mathcal{F}$ rappresenta un operatore di smoothing.

Tra i filtri comunemente utilizzati:

\begin{itemize}
    \item Media mobile (Moving Average),
    \item Filtro Gaussiano,
    \item Filtro esponenziale.
\end{itemize}

La scelta della finestra di smoothing influenza direttamente il trade-off tra stabilità e reattività del movimento.

\subsection{Moving Average}

Il filtro a media mobile centrata calcola, per il frame $t$, la media della traiettoria entro una finestra simmetrica di semi-ampiezza $w/2$:

\[
\tilde{C}_t = \frac{1}{|\mathcal{W}_t|} \sum_{i \in \mathcal{W}_t} C_i, \qquad \mathcal{W}_t = [\max(0,\, t-\tfrac{w}{2}),\; \min(N,\, t+\tfrac{w}{2})]
\]

L'implementazione scorre frame per frame calcolando la media dei valori nella finestra corrente, gestendo automaticamente i bordi con finestre di dimensione ridotta. Questo filtro attenua uniformemente tutte le frequenze al di sopra di $f_c \approx f_{fps}/(w)$.

\subsection{Filtro Gaussiano}

Il filtro gaussiano sostituisce il kernel uniforme con una finestra pesata:

\[
\tilde{C}_t = \sum_{i \in \mathcal{W}_t} h_i \cdot C_i, \qquad h_i \propto \exp\!\left(-\frac{(i-t)^2}{2\sigma^2}\right)
\]

dove il kernel $\{h_i\}$ viene ricalcolato e rinormalizzato per ogni frame per gestire correttamente i bordi della sequenza. Il parametro $\sigma$ (default: $w/6$) controlla la larghezza della gaussiana: valori più bassi producono meno smorzamento, valori più alti avvicinano il comportamento alla media mobile. Rispetto alla media mobile, il filtro gaussiano riduce le oscillazioni \emph{ringing} causate dalla discontinuità del kernel uniforme.

\subsection{Filtro Esponenziale}

Il filtro esponenziale (Exponential Moving Average, EMA) è un filtro IIR causale che processa la sequenza in un unico passaggio:

\[
\tilde{C}_t = \alpha \cdot C_t + (1 - \alpha) \cdot \tilde{C}_{t-1}, \qquad \alpha = \frac{2}{w + 1}
\]

A differenza dei filtri a finestra scorrevole, richiede memoria $O(1)$ ed è adatto all'elaborazione in streaming. Il coefficiente $\alpha$ è derivato dalla \texttt{smoothing\_window} con la relazione approssimata citata, ma è configurabile indipendentemente \texttt{exponential\_alpha}. Con $\alpha \to 1$ il filtro segue fedelmente l'originale (poco smoothing); con $\alpha \to 0$ reagisce lentamente alle variazioni (molto smoothing).

\section{Compensazione del Moto}

La trasformazione correttiva da applicare a ciascun frame è definita come:

\[
T_t^{corr} = \tilde{C}_t \circ C_t^{-1}
\]

Applicando tale trasformazione a ogni frame si ottiene:

\[
I_t^{stab} = T_t^{corr}(I_t)
\]

Poiché la compensazione comporta spostamenti dell'immagine, è necessario gestire le regioni fuori campo tramite:

\begin{itemize}
    \item cropping dinamico,
    \item padding con modalità di bordo,
    \item ridimensionamento globale.
\end{itemize}

La stabilizzazione finale è dunque il risultato della combinazione di:
\begin{enumerate}
    \item stima del moto,
    \item costruzione della traiettoria,
    \item filtraggio,
    \item compensazione geometrica.
\end{enumerate}
