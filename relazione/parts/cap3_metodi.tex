% =================================================
% CAPITOLO 3 – Metodi di Stabilizzazione Implementati
% =================================================

\chapter{Metodi di Stabilizzazione Implementati}

\section{Panoramica Generale}

Il sistema sviluppato integra tre differenti approcci per la stima del moto globale tra frame consecutivi:

\begin{itemize}
    \item Block-Based Motion Estimation (Block Matching) con metrica SAD/MAD,
    \item Optical Flow basato su corner detection Shi-Tomasi e tracciamento Lucas-Kanade,
    \item Feature Matching tramite descrittori ORB con Brute-Force Matcher.
\end{itemize}

Tutti i metodi condividono la medesima pipeline a due passate:

\begin{enumerate}
    \item \textbf{Primo passaggio}: stima del moto incrementale frame per frame e costruzione della traiettoria cumulativa.
    \item \textbf{Secondo passaggio}: applicazione del filtro di smoothing e compensazione geometrica per ogni frame.
\end{enumerate}

La differenza principale risiede nella modalità di estrazione delle corrispondenze e nella stima dei parametri di trasformazione globale.

\section{Block-Based Motion Estimation}

Il metodo Block Matching, implementato nella classe \texttt{MotionEstimator}, suddivide ciascun frame in blocchi non sovrapposti di dimensione fissa \texttt{block\_size}$\times$\texttt{block\_size} e ricerca il miglior match nel frame precedente entro una finestra di ricerca di raggio \texttt{search\_range}.

\subsection{Funzione di Costo}

Il matching è guidato da una funzione di costo pixel-wise. Sono implementate due metriche:

\begin{itemize}
    \item \textbf{SAD (Sum of Absolute Differences)}:
    \[
        D_{SAD}(B_t, B_{t-1}) = \sum_{i,j} \bigl|I_t(i,j) - I_{t-1}(i+\Delta x, j+\Delta y)\bigr|
    \]
    \item \textbf{MAD (Mean Absolute Difference)}:
    \[
        D_{MAD}(B_t, B_{t-1}) = \frac{1}{N^2}\sum_{i,j} \bigl|I_t(i,j) - I_{t-1}(i+\Delta x, j+\Delta y)\bigr|
    \]
\end{itemize}

Per ogni blocco si esegue una ricerca esaustiva su tutte le posizioni possibili nell'area di ricerca, ottenendo il vettore di spostamento locale $(\Delta x_k, \Delta y_k)$ e un punteggio di confidenza $c_k \in [0,1]$.

\subsection{Aggregazione del Moto Globale}

I vettori locali vengono prima filtrati tramite z-score per rimuovere gli outlier (vettori con z-score superiore a \texttt{outlier\_threshold} deviazioni standard su almeno una componente), quindi aggregati tramite uno dei tre metodi configurabili:

\begin{itemize}
    \item \textbf{Mediana} (\texttt{median}): robusta agli outlier residui.
    \item \textbf{Media} (\texttt{mean}): semplice e veloce.
    \item \textbf{Media pesata} (\texttt{weighted\_mean}): ogni vettore contribuisce proporzionalmente alla propria confidenza $c_k$.
\end{itemize}

La stima della componente di rotazione nel Block Matching puro restituisce $\Delta\theta = 0$ per default: il modello è essenzialmente traslazionale.

\subsection{Parametri Configurabili}

\begin{center}
\begin{tabular}{lll}
\toprule
Parametro & Default & Descrizione \\
\midrule
\texttt{block\_size} & 32 & Dimensione blocco (px) \\
\texttt{search\_range} & 12 & Raggio finestra di ricerca (px) \\
\texttt{metric} & \texttt{sad} & Metrica di similarità \\
\texttt{outlier\_threshold} & 2.0 & Soglia z-score per rimozione outlier \\
\texttt{aggregation\_method} & \texttt{median} & Metodo di aggregazione \\
\bottomrule
\end{tabular}
\end{center}

\section{Optical Flow – Shi-Tomasi + Lucas-Kanade}

Il metodo Optical Flow, attivato impostando \texttt{estimation\_method: optical\_flow} e \texttt{feature\_type: shi\_tomasi}, sfrutta la pipeline \texttt{goodFeaturesToTrack} + \texttt{calcOpticalFlowPyrLK} di OpenCV.

\subsection{Rilevamento dei Corner (Shi-Tomasi)}

La funzione \texttt{cv2.goodFeaturesToTrack} individua i punti salienti massimizzando il minimo autovalore della matrice di struttura locale~\cite{shi1994good}. I parametri chiave sono:

\begin{itemize}
    \item \texttt{maxCorners} (default: 1500): numero massimo di feature estratte per frame.
    \item \texttt{qualityLevel} (default: 0.002): soglia sul rapporto tra il minimo autovalore e il massimo globale.
    \item \texttt{minDistance} (default: 10\,px): distanza minima tra corner adiacenti.
\end{itemize}

\subsection{Tracciamento Lucas-Kanade Piramidale}

Lo spostamento di ciascun corner è stimato risolvendo il sistema lineare dell'ipotesi di costanza dell'intensità con \texttt{cv2.calcOpticalFlowPyrLK}~\cite{lucas1981iterative}:

\begin{itemize}
    \item \texttt{winSize} (default: $21\times21$): finestra di integrazione locale.
    \item \texttt{maxLevel} (default: 3): numero di livelli della piramide gaussiana, che consente di gestire spostamenti maggiori.
\end{itemize}

I punti il cui flag di stato è 0 (tracking non convergente) vengono scartati.

\subsection{Stima della Trasformazione Globale via RANSAC}

I punti tracciati con successo $\{(p_k, p'_k)\}$ vengono usati per stimare la trasformazione globale tramite RANSAC~\cite{fischler1981ransac}. Sono supportati tre modelli:

\begin{enumerate}
    \item \textbf{Partial} (\texttt{estimateAffinePartial2D}, 4 parametri): similitudine rigida con traslazione $(t_x, t_y)$, rotazione $\theta$ e scala uniforme $s$.
    \[
        M = s\begin{bmatrix}\cos\theta & -\sin\theta \\ \sin\theta & \cos\theta\end{bmatrix},\quad \mathbf{t}=\begin{bmatrix}t_x \\ t_y\end{bmatrix}
    \]
    \item \textbf{Affine} (\texttt{estimateAffine2D}, 6 parametri): trasformazione affine completa con 4 gradi di libertà aggiuntivi (scale differenziali e shear).
    \item \textbf{Homography} (\texttt{findHomography}, 8 parametri): trasformazione proiettiva completa $H \in \mathbb{R}^{3\times3}$.
\end{enumerate}

La soglia RANSAC (\texttt{ransac\_reproj\_threshold}, default: 6.0\,px) controlla toleranza agli inlier. L'angolo di rotazione viene estratto dalla matrice stimata tramite $\theta = \mathrm{atan2}(M_{10}, M_{00})$.

\section{ORB Feature Matching}

Il metodo ORB~\cite{rublee2011orb}, attivato con \texttt{feature\_type: orb}, sostituisce la pipeline LK con un detector/descriptor ORB (\emph{Oriented FAST and Rotated BRIEF}) seguito da un Brute-Force Matcher con distanza di Hamming.

\subsection{Rilevamento e Descrizione}

\texttt{cv2.ORB\_create} parametri configurabili:

\begin{center}
\begin{tabular}{lll}
\toprule
Parametro & Default & Descrizione \\
\midrule
\texttt{orb\_n\_features} & 1000 & Numero massimo di keypoint estratti \\
\texttt{orb\_scale\_factor} & 1.2 & Fattore di scala piramidale \\
\texttt{orb\_n\_levels} & 10 & Livelli della piramide \\
\texttt{orb\_edge\_threshold} & 31 & Bordo escluso dalla rilevazione \\
\bottomrule
\end{tabular}
\end{center}

\subsection{Matching e Lowe's Ratio Test}

Il matching è eseguito con \texttt{cv2.BFMatcher(cv2.NORM\_HAMMING, crossCheck=False)} applicando il kNN con $k=2$. Le corrispondenze ambigue vengono scartate tramite il \emph{Lowe's ratio test}~\cite{lowe2004distinctive}:

\[
    m.distance < r \cdot n.distance, \qquad r = \texttt{orb\_ratio\_threshold} = 0.75
\]

dove $m$ e $n$ sono rispettivamente il primo e il secondo nearest neighbor. Solo le corrispondenze che superano questo test vengono utilizzate per la stima RANSAC.

\subsection{Stima della Trasformazione Globale}

Identica all'approccio Optical Flow: i punti matchati $\{(p_k, p'_k)\}$ alimentano gli stessi tre stimatori RANSAC (Partial, Affine, Homography).

\section{Confronto Strutturale tra i Metodi}

\begin{table}[H]
\centering
\begin{tabular}{lccc}
\toprule
Caratteristica & Block Matching & Optical Flow & ORB \\
\midrule
Tipo di feature & Blocchi pixel & Corner Shi-Tomasi & Keypoint FAST \\
Descrittore & SAD/MAD & Flusso ottico LK & BRIEF binario \\
Modello trasformazione & Traslazione & Partial/Affine/Homog. & Partial/Affine/Homog. \\
Stima robusta & Z-score & RANSAC & RANSAC \\
Stima rotazione & No & Sì & Sì \\
Complessità & Bassa & Media & Media-Alta \\
Robustezza illuminazione & Bassa & Media & Alta \\
\bottomrule
\end{tabular}
\caption{Confronto strutturale tra i metodi implementati}
\end{table}
